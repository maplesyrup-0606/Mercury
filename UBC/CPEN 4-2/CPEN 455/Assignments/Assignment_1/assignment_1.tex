\documentclass{article}
\usepackage[utf8]{inputenc}
% \usepackage[paperheight=16cm, paperwidth=12cm,% Set the height and width of the paper
% includehead,
% nomarginpar,% We don't want any margin paragraphs
% textwidth=10cm,% Set \textwidth to 10cm
% headheight=10mm,% Set \headheight to 10mm
% ]{geometry}
\usepackage{geometry}
\usepackage{fancyhdr}
\usepackage{amsmath}
\usepackage{amssymb}
\usepackage{xspace}
\usepackage{bm}
\usepackage{xcolor}
\usepackage[colorlinks,linkcolor=blue]{hyperref}
\usepackage{graphicx}


\makeatletter
\DeclareRobustCommand\onedot{\futurelet\@let@token\@onedot}
\def\@onedot{\ifx\@let@token.\else.\null\fi\xspace}
\def\iid{\emph{i.i.d}\onedot} \def\IID{\emph{I.I.D}\onedot}
\def\eg{\emph{e.g}\onedot} \def\Eg{\emph{E.g}\onedot}
\def\ie{\emph{i.e}\onedot} \def\Ie{\emph{I.e}\onedot}
\def\cf{\emph{c.f}\onedot} \def\Cf{\emph{C.f}\onedot}
\def\etc{\emph{etc}\onedot} \def\vs{\emph{vs}\onedot}
\def\wrt{w.r.t\onedot} \def\dof{d.o.f\onedot}
\def\aka{\emph{a.k.a}\onedot}
\def\etal{\emph{et al}\onedot}
\makeatother

\newcommand{\important}[1]{{\color{blue}{\bf\sf #1}}}

\title{CPEN455: Deep Learning \\ Homework Template}
\author{Created by 
Qihang Zhang}
\date{Finished: 2025 Jan. 9}

\begin{document}

\pagestyle{fancy}
\fancyhead{} % clear all header fields
\fancyhead[L]{\textbf{UBC CPEN455 2024 Winter Term 2 }}
\fancyhead[R]{\textbf{Homework Template}}

\maketitle
\thispagestyle{fancy}


\section{Problem 1}

\section{Problem 2}
\noindent
\textbf{A 2.1}\\

\noindent
To insert inline equations, use $\frac{1}{1-p}$. To bold characters in equations, type $\mathbf{b}$. For script style letters, use $\mathcal{N}$.\\

\noindent
For a displayed equation, you can use

\begin{equation}
    P(k) = {N \choose k}p^{k}(1-p)^{N - k}
\end{equation}\\
Or you can also use 
$$
f'(x) = \lim_{\Delta x \to 0} \frac{f(x + \Delta x) - f(x)}{\Delta x}
$$\\
Also for series equations, you can use
\begin{align}
A &= 2 \int_{-r}^{r} \sqrt{r^2 - x^2} \, dx \\
  &= 2r^2 \int_{-\frac{\pi}{2}}^{\frac{\pi}{2}} \cos^2\theta \, d\theta \\
  &= \pi r^2 
\end{align}


\noindent
For matrices, format as follows:
\begin{equation}
    \begin{bmatrix}
    x_{11} & x_{12} & \cdots & x_{1n} \\
    \vdots & \vdots & & \vdots \\
    x_{m1} & x_{m2} & \cdots & x_{mn} \\
    \end{bmatrix}
\end{equation} \\

\noindent
For more math symbols, check \href{https://oeis.org/wiki/List_of_LaTeX_mathematical_symbols}{Wiki}, \href{https://www.cmor-faculty.rice.edu/~heinken/latex/symbols.pdf}{LATEX Mathematical Symbols}, Google, or ask Chat-GPT.\\

\noindent
\textbf{A 2.2}\\

If you want to insert a picture:

\begin{figure}[h]
\centering
%\includegraphics[width=0.5\textwidth]{your_picture.jpg}
\caption{Caption for the image.}
\label{fig:image1}
\end{figure}

\noindent
\textbf{A 2.3}\\

To highlight words in a different color, you can use \textcolor{blue}{textcolor} to turn something blue. You can also \important{define custom commands} for frequent use.

\end{document}
