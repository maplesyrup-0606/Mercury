\documentclass[11pt]{article}

\usepackage{amsmath}
\usepackage{amsthm}
\usepackage{amsfonts}
\usepackage[margin=1in]{geometry}

\newcommand{\matlab}{\textsc{Matlab\ }}

\title{CPSC 303, 2024/25, Term 2, Assignment 2}
\author{Released Wednesday, January 29, 2025 \\
Due Wednesday, February  12, 2025, 11:59pm}
\date{}
\begin{document}
\maketitle 
\thispagestyle{empty}
\noindent
{\em {\underline {Note}}: Questions 2 through 10 are taken from the textbook, and are provided below in full, for your convenience. At the beginning of each of them we state where they can be found  in the book. For example, ``AG 10.8'' means that we are referring to Ascher \& Greif, Chapter 10, Exercise 8.}

\begin{enumerate}

%%% Question 1
\item Consider the function $$f(x) = \sin (\alpha x), $$ where the values
of $\alpha$ to be considered in this question are
$$ \alpha = 0.1 \ ; \  \alpha=1\ ;  \
\alpha=12.$$ Suppose we wish to interpolate the function (for the
three different values of $\alpha$) over the interval $[0,\pi]$,
using the 11 data values $\{ (\frac{\pi j}{10},\sin (\frac{\alpha
\pi j}{10}) \},\ j=0,1,\dots,10.$

\begin{enumerate}
\item Write \matlab programs to generate approximations using
\begin{enumerate}
\item[(i)] a global 10th degree polynomial interpolant;
\item[(ii)] a not-a-knot cubic spline
\item[(iii)] a piecewise cubic Hermite polynomial
\item[(iv)] a piecewise linear polynomial
\item[(v)] a piecewise constant polynomial
\end{enumerate}
Either write your own code or use code provided with the book, and for (ii) and (iii) you may use the \matlab commands {\tt spline} and {\tt pchip} (check out {\tt pchip} carefully before you use it). We recommend that you write your own code -- your best way to learn.
Generate plots of the function and the approximating polynomials in
one graph (non-logarithmic scale, using {\tt plot}) and plots of the
error (logarithmic scale, using {\tt semilogy}), for the three
values of $\alpha$: a total of six graphs, with several curves on each of them. You may use the command
{\tt subplot} for a compact representation of your results.
\item Write down the error bounds for the five approaches. You may use information on bounds in the textbook or in other sources. (If your splint is not-a-knot, which is what the \matlab command {\tt spline} does, there is a constant $c$ in the error bound which you may ignore.) Comment on the quality of the approximations and the errors, and observe how well the error bounds predict the
actual errors. Specifically, make sure to explain how the value of
$\alpha$ (which determines how oscillatory the function is) makes an
effect.
\end{enumerate}

%%% Question 2
\item (AG 10.8)
A secret formula for eternal youth, $f(x)$, was discovered
by Dr. Quick, who has been working in our biotech company.
However, Dr. Quick has disappeared and is rumored to be
negotiating with a rival organization.

From the notes that Dr. Quick left behind
in his hasty exit it is clear that
$f(0) = 0$, $f(1) = 2$, and that $f[x_0,x_1,x_2] = 1$
for {\em any} three points
$x_0, x_1, x_2$.
Find $f(x)$.

\item (AG 10.9)
     Joe had decided to buy stocks of a particularly promising
      Internet company. The price per share was $\$100$, and Joe
      subsequently recorded the stock price at the end of each
      week. With the abscissae measured in days, the following data were
      acquired: \\
      $(0,100), (7,98), (14,101), (21,50), (28,51), (35,50)$.


      In attempting to analyze what happened, it was desired to
      approximately evaluate the stock price a few days before
      the crash.
  \begin{enumerate}
    \item Pass a linear interpolant through the points with
      abscissae $7$ and $14$. Then add to this data set the value at $0$
      and (separately) the value at $21$ to obtain two quadratic
      interpolants. Evaluate all three interpolants at $x=12$.
      Which one do you think is the most accurate? Explain.
    \item Plot the two quadratic interpolants above, together
      with the data (without a broken line passing through the data)
      over the interval $[0,21]$.
      What are your observations?
  \end{enumerate}



%%% Question 4
\item (AG 10.19) Interpolate the Runge function of Example~10.6 at Chebyshev points
for $n$ from 10 to 170 in increments of 10.
Calculate the maximum interpolation error on the uniform evaluation mesh {\tt x = -1:.001:1}
and plot the error vs. polynomial degree as in Figure~10.8 using {\tt semilogy}.
Observe spectral accuracy.


%%% Question 5
\item (AG 10.22) For some function $f$, you have a table of extended divided
  differences of the form
  \begin{center}
    \begin{tabular}[h]{c|c|cccc} 
$i$ & $z_i$ & ${\quad}f[\cdot]{\quad}$ & ${\quad}f[\cdot,\cdot]{\quad}$ &
$f[\cdot,\cdot,\cdot]$ &
$f[\cdot, \cdot, \cdot, \cdot]$ \\ \hline
0 & 5.0   & $f[z_0]$ & & & \\ 
1 & 5.0   & $f[z_1]$ & $f[z_0,z_1]$ & & \\ 
2 & 6.0 & 4.0 & 5.0 & $-3.0$ & \\ 
3 & 4.0 & 2.0 &$ f[z_2,z_3]$ & $f[z_1,z_2,z_3]$ & $f[z_0,z_1,z_2,z_3]$  \\
    \end{tabular}
  \end{center}
  Fill in the unknown entries in the table. 

%%% Question 6
\item (AG 10.23)  For the data in Exercise~10.22 (Question 5 in this assignment)
 what is the osculating polynomial $p_2(x)$ of degree
     at most 2 that satisfies
     \[ p_2(5.0)=f(5.0),\ p_2'(5.0)=f'(5.0),\   p_2(6.0)=f(6.0) ? \]


%%% Question 7
\item (AG 11.3) Let $f \in C^3[a,b]$ be given at equidistant points $x_i = a + ih, \; i = 0, 1, \ldots , n$,
where $nh = b-a$. Assume further that $f'(a)$ is given as well.
\begin{enumerate}
\item Construct an algorithm for $C^1$ piecewise quadratic interpolation of the given values.
Thus, the interpolating function is written as
\begin{eqnarray*} 
v(x) = s_i (x) =  a_i + b_i (x-x_i) + c_i (x-x_i)^2, \quad x_i \leq x \leq x_{i+1},
\end{eqnarray*}
for $i=0, \ldots , n-1$,
and your job is to specify an algorithm for determining the $3n$
coefficients $a_i, \ b_i$ and $c_i$.
\item
How accurate do you expect this approximation to be as a function of $h$? Justify.
\end{enumerate}

%%% Question 8
\item (AG 11.4)
Verify that the Hermite cubic interpolating $f(x)$ and its derivative
at the points $t_i$ and $t_{i+1}$ can be written explicitly as
\begin{eqnarray*}
s_i(x) &=& f_i + \left( h_if_i' \right)\tau + \Big( 3(f_{i+1}-f_i) - h_i(f_{i+1}'+2f_i') \Big)\tau^2\\
&+& \Big( h_i(f_{i+1}'+f_i') - 2(f_{i+1} - f_i) \Big) \tau^3,
\end{eqnarray*}
where $h_i = t_{i+1}-t_i$, $f_i=f(t_i),~f_i' = f'(t_i),~
f_{i+1}=f(t_{i+1}),~f_{i+1}' = f'(t_{i+1}),$ and
$\tau = \frac{x-t_i}{h_i}$.


%%% Question 9
\item (AG 11.6, with a minor revision of part (d))
The {\em gamma function} is defined by
\begin{eqnarray*}
\Gamma(x)= \int_0^{\infty} t^{x-1} e^{-t} dt, \ \ x>0.
\end{eqnarray*}
It is known that  for integer numbers the function has the value
\begin{eqnarray*}
\Gamma(n)=(n-1)! = 1 \cdot 2 \cdot 3 \cdots (n-1).
\end{eqnarray*}
(We define $0!=1$.)
Thus, for example,  $(1,1),(2,1),(3,2),(4,6),(5,24)$
can be used as data points for an interpolating polynomial.
\begin{enumerate}
\item Write a \matlab script that  computes  the
polynomial interpolant of degree four
that passes through the above five data points.
\item Write a program that computes a cubic spline to interpolate
the same data. (You may use {\sc Matlab}'s {\tt spline}, or your own code.)
\item Plot the two interpolants you found on the same graph, along with
a plot of the gamma function itself, which can be produced using the
\matlab command {\tt gamma}.
\item Plot the errors in the two interpolants on the same graph. 
What are your observations? You may separate your observations into two parts: the errors for smaller values of $x$, e.g., on the interval $[0,3]$, and the errors for larger values, say on the interval $[3,5]$.
\end{enumerate}

%%% Question 10
\item (AG 11.15)
Consider interpolating the data $(x_0,y_0), \ldots , (x_6,y_6)$
given by
\begin{center}
          \begin{tabular}{|c|c|c|c|c|c|c|c|}
            x & 0.1 & 0.15 & 0.2 & 0.3 & 0.35 & 0.5 & 0.75 \\ \hline
            y & 3.0 & 2.0 &  1.2 & 2.1 & 2.0  & 2.5 & 2.5
          \end{tabular}
\end{center}
Construct the five interpolants specified below (you may use available software for this),
evaluate them at the points $0.05:0.01:0.8$, plot and comment
on their respective properties.
\begin{enumerate}
\item A polynomial interpolant.
\item A cubic spline interpolant.
\item The interpolant
\[   v(x) = \sum_{j=0}^n c_j \phi_j(x) = c_0 \phi_0(x) + \cdots + c_n
  \phi_n(x), \]
where $n = 7$, $\phi_0(x) \equiv 1$, 
\[ \phi_j (x) = \sqrt{ (x-x_{j-1})^2 + \varepsilon^2} - \varepsilon,
\quad j = 1, \ldots , n. \]
In addition to the $n$ interpolation requirements, the condition
\[ c_0 = -\sum_{j=1}^n c_j \]
is imposed.
Construct this interpolant with (i) $\varepsilon = 0.1$, (ii) $\varepsilon = 0.01$
and $\varepsilon = 0.001$.
Make as many observations as you can.
What will happen if we let $\varepsilon \rightarrow 0$? 
\end{enumerate}

 \end{enumerate}

\end{document}
